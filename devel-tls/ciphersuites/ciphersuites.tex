\chapter{Ciphersuites}

A ciphersuite is a collection of cryptographic primitives (e.g. algorithms for
encryption; algorithms for digital signatures etc) that can be used
together in the SSL/TLS protocol. During the intial message from the client
(the ``ClientHello'') a list of ciphersuites that the client supports is
provided. The server will then select one and communicate back to the client
which one it wants to use in its response (the ``ServerHello'' message).

OpenSSL 1.1.0 supports many different ciphersuites. A full list of all the
supported ciphersuites is available by issuing the following command:

\begin{verbatim}
$ openssl ciphers -v ALL:eNULL
\end{verbatim}

Some sample output is provided in listing \ref{lst:ciphers-extract}. The output
from this command will tell you, for each supported ciphersuite:

\begin{itemize}
\item The ciphersuite name.
\item The earliest protocol version that the ciphersuite can be used with.
\item The key exchange (Kx) algorithm that it uses.
\item The authentication (Au) algorithm that it uses.
\item The encryption (Enc) algorithm that it uses.
\item The message digest algorithm used within the Message Authentication Code
(Mac) algorithm. A value of AEAD here indicates that there is no separate MAC,
but the encryption algorithm provides integrity protection.
\end{itemize}

\section{Ciphersuite naming}

\todo{stuff here}

\section{Configuring the available ciphersuites}

Ciphersuites are configured via a \emph{cipher list}, which determines:
\begin{enumerate}
\item which ciphersuites are enabled for use; and
\item what the order of preference is for those ciphersuites.
\end{enumerate}

The cipher list is constructed from a sequence of \emph{cipher strings} (also
known as \emph{aliases}) separated by colons ``\verb!:!''. Each cipher string
may additionally be modified by a preceding operator to change its meaning
within the definition of the cipher list.

A cipher string (or alias) simply represents an ordered set of ciphersuites
(with the exception of the ``special'' cipher strings explained below).
There are four types of cipher string:
\begin{enumerate}
\item Simple cipher strings that represent a set containing exactly one
ciphersuite. The cipher string name is simply the name of the ciphersuite. Examples
include ``\verb!AES128-SHA!'', ``\verb!ECDHE-RSA-AES256-SHA!'' and
``\verb!PSK-AES128-GCM-SHA256!''.
\item Complex cipher strings that represent a whole class of ciphersuites. For
example ``\verb!AES128!'' represents the set of all ciphersuites that use AES
128 bit encryption; ``\verb!ECDHE!'' represents the set of all authenticated
ciphersuites that use ephemeral ECDH key agreement; and ``\verb!eNULL!''
represents the set of all ciphersuites that provide no encryption at all. Don't
worry about the specific meanings of what these represent. It will be covered in
more detail in the rest of this chapter.
\item Composite cipher strings that represent the intersection of two or more
simple or complex cipher strings. Cipher strings of this type are formed by
listing each of the constituent cipher strings in turn separated by a
``\verb!+!'' character. For example ``\verb!AES128+ECDHE!'' represents the set
of ciphersuites that use AES 128 bit encryption and also use ephemeral ECDH key
agreement.
\item ``Special'' cipher strings. These have a special modifying effect on a
cipher list that is being constructed. Examples include ``\verb!@STRENGTH!'' and
``\verb!@SECLEVEL!''. These will be explained in more detail in the sections
below.
\end{enumerate}

The cipher list is defined by combining cipher strings together. By default
adding cipher strings onto the end of a cipher list (with each cipher string
separated by a colon, ``\verb!:!''), will add the ordered set of ciphers
covered by that cipher string onto the end of the list built so far. So, for
example, the cipher list ``\verb!AES256:AES128!'' will first add all
ciphersuites based on AES 256 bit encryption, and then add all of the
ciphersuites based on AES 128 bit encryption, \emph{in that order}. When adding
a cipher string onto a cipher list, if a ciphersuite contained within the
cipher string is already in the list then that ciphersuite is ignored, i.e. it
is not added again. For example the cipher list
``\verb!AES128-CCM:AES128-SHA:AES128-CCM!'' will contain exactly two
ciphersuites, ``\verb!AES128-CCM!'' and ``\verb!AES128-SHA!'' in that order. The
additional ``\verb!AES128-CCM! on the end has no effect because it is already in
the cipher list.

Modifiers can be added to the front of cipher strings to change the effect that
they have on the cipher list that is being constructed. The available modifiers
are:
\begin{itemize}
\item \texttt{+} - This has the effect of moving matching ciphersuites that are
\emph{already} in the cipher list to the end. It does not add new ciphersuites.
For example the cipher list ``\texttt{AES128:+RSA}'' will only contain
ciphersuites using AES 128 bit encryption, but any of those that also use RSA
authentication will be at the end of the list.
\item \texttt{-} - Removes the ciphersuites in the cipher string from the
cipher list. For example the cipher list ``\texttt{RSA:-eNULL}'' will contain
the list of all ciphersuites that use RSA authentication \emph{except} those
that do not use encryption. Ciphersuites removed from the list in this way can
be added back in again by later cipher strings.
\item \texttt{!} - Permanently removes the ciphersuites in the cipher string
from the cipher list. For example the cipher list ``\texttt{RSA:!eNULL}'' will
contain the list of all ciphersuites that use RSA authentication except those
that do no use encryption. No cipher strings added to the end of this list will
ever add in any ciphersuites that do not use encryption.
\end{itemize}

Be careful not to confuse the two different uses of the ``\verb!+!'' operator.
Dependant on the context, this can either be used to construct a composite
cipher string, or to move matching ciphersuites to the end of the list. For
example the cipher list ``\verb!AES128+RSA!'' contains all ciphersuites that use
AES 128 bit encryption \emph{and} RSA authentication. In this context the
``\verb!+!'' is being used to form a composite cipher string. However the cipher
list ``\verb!AES128:+RSA!'' contains all ciphersuites that use AES 128 bit
encryption, with any of those that also use RSA authentication moved to the end
of the list. In this context the ``\verb!+!'' is being used as a cipher string
modifier.

\subsection{Special cipher strings}

There are two ``special'' cipher strings. The first is ``\verb!@STRENGTH!''.
This has the effect of sorting the cipher list that has been built so far by
order of the number of security bits each ciphersuite uses for its encryption
key. Mostly the number of security bits in the key and the total number of bits
in the key is the same thing. So for example AES 128 has a total key length of
128 bits, and all of those contribute to the number of security bits, i.e. AES
128 has 128 security bits. The main exceptions to this rule are DES and
Triple-DES (also sometimes referred to in OpenSSL ciphersuite names as
``\verb!3DES-EDE!'' or ``\verb!DES-CBC3!''). DES has a total key length of 64
bits, but only 56 of those contribute towards security (the remainder are parity
bits). Therefore the number of security bits in DES is 56. Triple-DES has a
total key length of 192 bits. However the number of security bits is
112\footnote{ The DES and Triple-DES ciphersuites are not available in a default
build of OpenSSL. In order to use them OpenSSL must have been configured at
build time with the ``enable-weak-ssl-ciphers'' option}.

Note: the ``\verb!@STRENGTH!'' cipher string can be useful but it should be
used with care. There are many factors that contribute towards the overall
security provided by any given ciphersuite. It is not just about how many
security bits the encryption key provides. In addition you may also wish to
consider other factors such as performance, perfect forward secrecy (see
section \todo{add ref here}) and interoperability.

The other ``special'' cipher string is ``\verb!@SECLEVEL!''. This sets the
\emph{security level} to be applied. The security level can be set to any value
between 0 and 5 (inclusive). By default the security level is 1. The security
level can also be set programmatically by calling
\verb!SSL_CTX_set_security_level! or \verb!SSL_set_security_level!. Dependant
on its value, the security level can have the impact of preventing certain
ciphersuites from ever being selected. Note that the ``\verb!openssl ciphers!''
command will still show ciphers that are below the current security level
unless you additionally use the ``\verb!-s!'' argument. The security levels
have the following meanings:
\begin{itemize}
\item Level 0: No restrictions
\item Level 1: 80 bits of security. Encryption algorithms with less than 80
security bits will be excluded, as will RSA, DSA and DH keys shorter than 1024
bits and ECC\footnote{Elliptic Curve Cryptography (ECC) algorithms are a class
of cryptographic algorithms based on the mathematics of elliptic curves. In
OpenSSL this includes ECDH/ECDHE and ECDSA.} keys less than 160 bits in length.
Also prohibited is any ciphersuite based on MD5 for its MAC.
\item Level 2: 112 bits of security.  Encryption algorithms with less than 112
security bits will be excluded, as will RSA, DSA and DH keys shorter than 2048
bits and ECC keys less than 224 bits in length. Additionally MD5 based MACs,
SSLv3 and compression are prohibited.
\item Level 3: 128 bits of security.  Encryption algorithms with less than 128
security bits will be excluded, as will RSA, DSA and DH keys shorter than 3072
bits and ECC keys less than 256 bits in length. Additionally MD5, SSLv3,
TLSv1.0, compression and session tickets are prohibited.
\item Level 4: 192 bits of security. Encryption algorithms with less than 192
security bits will be excluded, as will RSA, DSA and DH keys shorter than 7680
bits and ECC keys less than 384 bits in length. Additionally MD5 or SHA1 based
MACs, SSLv3, TLSv1.0, TLSv1.1, compression and session tickets are prohibited.
\item Level 5: 256 bits of security. Encryption algorithms with less than 256
security bits will be excluded, as will RSA, DSA and DH keys shorter than 15360
bits and ECC keys less than 512 bits in length. Additionally MD5 or SHA1 based
MACs, SSLv3, TLSv1.0, TLSv1.1, compression and session tickets are prohibited.
\end{itemize}

An example of a cipher list using this is ``\verb!AES256:AES128:@SECLEVEL=3!''.
This enables all ciphersuites using AES with 256 or 128 bit keys but will
prevent any of those based on RSA, DSA or DH from using keys less than 3072
bits (amongst the other restrictions listed above).

\section{Ciphersuite selection}

Having built a cipher list of possible ciphers OpenSSL will ignore any that it
is not currently capable of supporting. For example the cipher list could
include ciphersuites that are not allowed due to the current security level. If
OpenSSL is being used as a client then the it will build its ClientHello to
contain a list of all those ciphers in the current cipher list and that are
currently allowed. They will be in the order of preference as defined in the
cipher list.

The server will select which of the ciphersuites offered by the client will be
used. If OpenSSL is used as a server then it can use one of two possible
algorithms for selecting the ciphersuite:
\begin{enumerate}
\item Client preference
\item Server preference
\end{enumerate}

The default is to use ``client preference''. With this approach the server will
choose the first ciphersuite from the list presented by the client which is
also in the server's cipher list.

If the server is using ``server preference'' then the server will choose the
first cipher from its own cipher list which is also in the client's cipher
list. In order to enable ``server preference'' you should call one of the
\verb!SSL_CTX_set_options()! or \verb!SSL_set_options()! functions and set
the \verb!SSL_OP_CIPHER_SERVER_PREFERENCE! option.

Some ciphersuites may need some additional configuration before they become
available to clients or servers \emph{even if} they are present in the cipher
list. For example a server will never select ECDSA based ciphersuites if
it has only been set up with an RSA certificate and key. Similarly it will never
select an RSA ciphersuite if it only has an ECDSA certificate and key. It
is also possible to set up a server with multiple certificates of different
types to enable more ciphersuites if desired.

\section{Key Exchange}

All ciphersuites define an algorithm for securely exchanging cryptographic
keys. Clearly it is important for such an algorithm that it is not possible for
any eavesdropper to learn the cryptographic keys that are being exchanged.
There are a number of different algorithms that can be used to solve this
problem. OpenSSL supports the following algorithms:
\begin{itemize}
\item RSA
\item DH (including DHE)
\item ECDH (including ECDHE)
\item SRP
\item PSK (including plain PSK as well as DHEPSK, ECDHEPSK and RSAPSK variants)
\end{itemize}

\subsection{RSA}

RSA is a public-key encryption system\footnote{Unlike symmetric encryption
systems where both parties have the same key in a public-key encryption system
there are two keys: a private key and a public key. These two keys are
mathematically related to each other but it is not possible to derive the
private key if you only know the public key. One party (Alice) generates a
private/public key pair and publishes the public key (normally in the form of a
certificate). Anyone may see the public key but the private key must be kept
secret. Another party (Bob) can encrypt messages destined for Alice using
her public key. Alice decrypts them using the private key. No one else is able
to decrypt them because the private key is secret.} that can be used for both
key exchange and authentication (see section \todo{ref here})\footnote{RSA is
not suitable for the main encryption algorithm as it is significantly less
performant than most symmetric encryption systems}. When used for key exchange,
the client chooses a secret value and then encrypts it using the public key
from the server's certificate. The server can then decrypt this value using its
private key, so both parties now know the same secret.

To enable this key exchange method for OpenSSL servers it is necessary to
provide an RSA based certificate and associated private key. Server
applications should call \verb!SSL_CTX_use_certificate_file()!,
\verb!SSL_CTX_use_PrivateKey_file()! and \verb!SSL_CTX_check_private_key()! as
described in section \ref{sec:get-start-setup-ssl-ctx}. There are also variants
of these functions to set this up on a per SSL object basis; to load
certificates from raw binary (ASN.1) data; or to load them from an OpenSSL
\verb!X509! object.

\subsection{DH}

The Diffie-Hellman (DH) key exchange algorithm works in a very different way to
RSA. In this approach both client and server create a private and a public
value that are mathematically related. It is not possible to derive the private
value from the public value. They both exchange their respective public values
and perform a mathematical operation to combine their own private value with
the public value received from their peer. The result will be the same for both
parties and is a shared secret. An eavesdropper will be unable to derive the
shared secret because they will not have access to either of the private
values.

